\documentclass[fleqn]{article}

%%%%%%%%%%%%%%%%%%%%% Pre-document %%%%%%%%%%%%%%%%%%%%%
\usepackage{fancyhdr}
\usepackage{titlesec}
\usepackage{float}
\usepackage{array}
\usepackage{nicematrix}
\usepackage{multicol}
\usepackage{enumitem}
\usepackage{listings}
\usepackage{xcolor}
\usepackage{scrextend}
\usepackage{bm}
\usepackage{circuitikz}

\lstdefinelanguage[RISC-V]{Assembler} {
  alsoletter={.},
  alsodigit={0x},
  morekeywords=[1]{ % instructions
    lw, sw,
    sll, slli,
    add, addi, sub,
    xor, xori, or, ori, and, andi,
    beq, bne, blt, bge, bltu, bgeu,
    j, jr, jal, jalr, ret,
  },
  morekeywords=[2]{ % registers
    x0, x1, x2, x3, x4, x5, x6, x7, x7, x8, x9, x10,
    x11, x12, x13, x14, x15, x16, x17, x18, x19, x20, 
    x21, x22, x23, x24, x25, x26, x27, x28, x29, x30, x31 },
  morecomment=[l]{;},   % mark ; as line comment start
  morecomment=[l]{\#},  % as well as #
}

% define some colors
\definecolor{pink}{rgb}{204, 0, 68}
\definecolor{blue}{rgb}{77, 148, 255}

\lstset{
  % listings sonderzeichen (for german weirdness)
  literate={ö}{{\"o}}1
           {ä}{{\"a}}1
           {ü}{{\"u}}1,
  basicstyle=\ttfamily,
  breaklines=true,
  commentstyle=\color{gray!50!black},
  keywordstyle=[1]\color{pink},
  % keywordstyle=[2]\color{blue}, 
  stringstyle=\color{mauve},                    % strings are from the telekom
  language=[RISC-V]Assembler,                   % all code is RISC-V
  tabsize=4,                                    % indent tabs with 4 spaces
  showstringspaces=false                        % do not replace spaces with weird underlines
}


\setlength{\parindent}{0pt} % Remove auto paragraph indents
\renewcommand\labelitemi{{\boldmath$\cdot$}}
\setlength{\mathindent}{0pt}  

% Get rid of those big margins
\usepackage[margin=1in]{geometry}
\newlength\titleindent
\setlength\titleindent{2cm}

\begin{document}

\pagestyle{fancy}
% Header
\fancyhead{}
\fancyhead[L]{Stephanie L'Heureux}
\fancyhead[R]{\thepage}
% No page numbers for footer
\fancyfoot{}

\begin{center}
    \Large{\textbf{Appendix A Exercises}}\\
\end{center}
\vspace{0.25in}

\textbf{A.1} [10] \textlangle A.2\textrangle \; In addition to the basic laws we discussed in this section, there are two important theorems, called DeMorgan's theorems:
$$\overline{A + B} = \overline{A} \cdot \overline{B} \text{ and } \overline{A \cdot B} = \overline{A} + \overline{B}$$
Prove DeMorgan's theorems with a truth table of the form:

\begin{table}[H]
\centering
    \begin{tabular}{cc|cc|cc|cc}
    \bm{$A$} & \bm{$B$} & $\bm{\overline{A}}$ & $\bm{\overline{B}}$ & $\bm{\overline{A + B}}$ & $\bm{\overline{A} \cdot \overline{B}}$ & $\bm{\overline{A \cdot B}}$ & $\bm{\overline{A} + \overline{B}}$ \\\hline
    0 & 0 & 1 & 1 & 1 & 1 & 1 & 1 \\
    0 & 1 & 1 & 0 & 0 & 0 & 1 & 1 \\
    1 & 0 & 0 & 1 & 0 & 0 & 1 & 1 \\
    1 & 1 & 0 & 0 & 0 & 0 & 0 & 0
    \end{tabular}
\caption{DeMorgan's Theorems}
\end{table}

\vspace{0.125in}
\begin{addmargin}[0.15cm]{0cm}
The table above is given in the text. Note that columns $\overline{A + B}$  and $\overline{A} \cdot \overline{B}$ contain the same values and $\overline{A \cdot B}$ and columns $\overline{A} + \overline{B}$ also contain the same values, thus proving DeMorgan's Theorems.
\end{addmargin}

\vspace{0.5in}
\textbf{A.3} [10] \textlangle A.2\textrangle \; Show that there are $2^n$ entries in a truth table for a function with $n$ inputs.
\vspace{0.125in}
\begin{addmargin}[0.15cm]{0cm}
    Let boolean function $f$ with $n$ inputs be written as:
    $$f(x_1, x_2,\; ... \; , x_{n-1}, x_n)$$
    Each input $x_i$ can take on one of two values, 0 or 1. The total number of input combinations is given by the product of the possibilities for each of the $n$ inputs:
    $$\prod_{i=1}^{n} 2 = 2 \times 2 \times ... \times 2 \times 2 = 2^n$$
    Therefore, the number of different input combinations is $2^n$, and since each row in a truth table corresponds to one unique combination of inputs, there are $2^n$ entries in a truth table for a function with $n$ inputs.
    
    \vspace{0.125in}
    Example where $n = 3$:
    \begin{table}[H]
\centering
    \begin{tabular}{r|r r r}
        \multicolumn{1}{l}{} & \multicolumn{1}{c}{$\bm{x_1}$} & \multicolumn{1}{c}{$\bm{x_2}$} & \multicolumn{1}{c}{$\bm{x_3}$} \\ \cline{2-4} 
        \multicolumn{1}{r|}{\textbf{1}} & 0 & 0 & 0 \\
        \multicolumn{1}{r|}{\textbf{2}} & 0 & 0 & 1 \\
        \multicolumn{1}{r|}{\textbf{3}} & 0 & 1 & 0 \\
        \multicolumn{1}{r|}{\textbf{4}} & 0 & 1 & 1 \\
        \multicolumn{1}{r|}{\textbf{5}} & 1 & 0 & 0 \\
        \multicolumn{1}{r|}{\textbf{6}} & 1 & 0 & 1 \\
        \multicolumn{1}{r|}{\textbf{7}} & 1 & 1 & 0 \\
        \multicolumn{1}{r|}{\textbf{8}} & 1 & 1 & 1 \\
    \end{tabular}
    \caption{Truth table with $n = 3$ inputs}
\end{table}

    Table contains $2^3 = 8$ rows.
\end{addmargin}

\pagebreak

\textbf{A.5} [15] \textlangle A.2\textrangle \; Prove that the NOR gate is universal by showing how to build the AND, OR, and NOT functions using a two-input NOR gate.

\begin{table}[H]
\centering
\begin{tabular}{cc|c}
    \textbf{a} & \textbf{b} & \textbf{NOR} \\ \hline
    0 & 0 & 1 \\ 
    0 & 1 & 0 \\ 
    1 & 0 & 0 \\ 
    1 & 1 & 0 \\ 
\end{tabular}
\caption{NOR gate}
\end{table}


\begin{multicols}{3}
    % NOT
    \begin{center}
    \textbf{NOT}
    \vspace{0.25 in}
        
    \begin{circuitikz}
        \node[nor port, number inputs=2] (NOR) at (1, 0) {};
        % Connect inputs for NOR
        \path (NOR.in 1) -- (NOR.in 2)
        coordinate[pos=0.5] (NOR_in);
        \draw (NOR_in) -- ++(-0.55, 0);
        \node[left] at (-1, 0) {$A$};
        \draw (NOR.in 1) -- (NOR.in 2);
        % Output
        \draw (NOR.out) -- node[above]{$\overline{A}$} ($(NOR) + (1, 0)$);
    \end{circuitikz}
    \end{center}
    By the truth table for a NOR gate (Table 3), when both inputs $A$ and $B$ are 0, the output is 1, and when both inputs are 1, the output is 0. This behavior mirrors that of a NOT operation. A NOT gate can therefore be created by connecting both inputs of a NOR gate.

    \columnbreak
    % OR
    \begin{center}
    \textbf{OR}
    \vspace{0.25 in}
    
    \begin{circuitikz}
        \node[nor port, number inputs=2] (NOR_1) at (0, 0) {};
        \node[nor port, number inputs=2] (NOR_2) at (2, 0) {};
        % Inputs for NOR 1
        \node[left] at (NOR_1.in 1) {$A$};
        \node[left] at (NOR_1.in 2) {$B$};
        % Connect inputs for NOR 2
        \draw (NOR_2.in 1) -- (NOR_2.in 2);
        % Connect NOR_1 and NOR_2
        \path (NOR_2.in 1) -- (NOR_2.in 2)
        coordinate[pos=0.5] (NOR_2_in);
        \draw (NOR_1.out) -- (NOR_2_in);
        % Output
        \draw (NOR_2.out) -- node[above]{$A + B$} ($(NOR_2) + (1, 0)$);
    \end{circuitikz}
    \end{center}
    The NOR gate produces the inverse of the OR operation. By applying a NOT operation (constructed using NOR gates) to the output of a NOR gate, the OR operation is implemented.
    
    \columnbreak
    \begin{center}
    \textbf{AND}
    \vspace{0.25 in}

    \begin{circuitikz}
    % NOT
    \node[nor port, number inputs=2] (NOR_1) at (1, 0) {};
    \path (NOR_1.in 1) -- (NOR_1.in 2)
    coordinate[pos=0.5] (NOR_1_in);
    \draw (NOR_1_in) -- ++(-0.55, 0);
    \node[left] at (-1, 0) {$A$};
    \draw (NOR_1.in 1) -- (NOR_1.in 2);
    \draw (NOR_1.out) -- ($(NOR_1) + (0.5, 0)$);

    % NOT
    \node[nor port, number inputs=2] (NOR_2) at (1, -1.5) {};
    \path (NOR_2.in 1) -- (NOR_2.in 2)
    coordinate[pos=0.5] (NOR_2_in);
    \draw (NOR_2_in) -- ++(-0.55, 0);
    \node[left] at (-1, -1.5) {$B$};
    \draw (NOR_2.in 1) -- (NOR_2.in 2);
    \draw (NOR_2.out) -- ($(NOR_2) + (0.5, 0)$);

    % NOR
    \node[nor port, number inputs=2] (NOR_3) at (3, -0.75) {};
    \draw ($(NOR_1) + (0.5, 0)$) -- (NOR_3.in 1);
    \draw ($(NOR_2) + (0.5, 0)$) -- (NOR_3.in 2);
    \draw (NOR_3.out) -- node[above]{$A \cdot B$} ($(NOR_3) + (1, 0)$);
\end{circuitikz}
\end{center}

$\overline{\overline{A} + \overline{B}} = \overline{\overline{A}} \cdot \overline{\overline{B}} = A \cdot B$ 
    
By DeMorgan's Theorem ($\overline{A} \cdot \overline{B} = \overline{A + B}$)
\end{multicols}


\vspace{0.5in}
\textbf{A.7} [10] \textlangle A.2, A.3\textrangle \; Construct the truth table for a four-input odd-parity function (see page A-65 for a description of parity).
\vspace{0.125in}
\begin{addmargin}[0.15cm]{0cm}
\begin{table}[H]
\centering
\begin{tabular}{cccc|c}
\textbf{A} & \textbf{B} & \textbf{C} & \textbf{D} & \textbf{Parity} \\ \hline
0 & 0 & 0 & 0 & 0 \\
0 & 0 & 0 & 1 & 1 \\
0 & 0 & 1 & 0 & 1 \\
0 & 0 & 1 & 1 & 0 \\
0 & 1 & 0 & 0 & 1 \\
0 & 1 & 0 & 1 & 0 \\
0 & 1 & 1 & 0 & 0 \\
0 & 1 & 1 & 1 & 1 \\
1 & 0 & 0 & 0 & 1 \\
1 & 0 & 0 & 1 & 0 \\
1 & 0 & 1 & 0 & 0 \\
1 & 0 & 1 & 1 & 1 \\
1 & 1 & 0 & 0 & 0 \\
1 & 1 & 0 & 1 & 1 \\
1 & 1 & 1 & 0 & 1 \\
1 & 1 & 1 & 1 & 0
\end{tabular}
\caption{Four-input odd-parity function}
\end{table}
\end{addmargin}

\vspace{0.5in}

\textbf{A.17} [5] \textlangle A.2, A.3\textrangle \; Show a truth table for a multiplexor (inputs A, B, and S; output C), using don't cares to simplify the table where possible.
\vspace{0.125in}

\begin{multicols}{3}
\begin{table}[H]
\centering
\begin{tabular}{ccc|c}
\textbf{A} & \textbf{B} & \textbf{S} & \textbf{C} \\ \hline
0 & 0 & 0 & 0 \\ 
0 & 0 & 1 & 0 \\
0 & 1 & 0 & 0 \\
0 & 1 & 1 & 1 \\
1 & 0 & 0 & 1 \\
1 & 0 & 1 & 0 \\
1 & 1 & 0 & 1 \\
1 & 1 & 1 & 1
\end{tabular}
% \caption{2-1 MUX}
\end{table}

\columnbreak

\begin{table}[H]
    \centering
    \begin{tabular}{ccc|c}
    \textbf{A} & \textbf{B} & \textbf{S} & \textbf{C} \\ \hline
    0 & X & 0 & 0 \\
    X & 0 & 1 & 0 \\
    1 & X & 0 & 1 \\
    X & 1 & 1 & 1
    \end{tabular}
    \end{table}

\columnbreak

\begin{table}[H]
\centering
\begin{tabular}{c|c}
\textbf{S} & \textbf{C} \\ \hline
0 & A \\
1 & B
\end{tabular}
% \caption{2-1 MUX simplified}
\end{table}
\end{multicols}


\begin{multicols}{3}
\begin{center}Table 5: 2-1 MUX\end{center}
\columnbreak
\begin{center}Table 6: 2-1 MUX with \\don't cares\end{center}
\columnbreak
\begin{center}Table 7: 2-1 MUX simplified\end{center}
\end{multicols}

\end{document}