\documentclass[fleqn]{article}

%%%%%%%%%%%%%%%%%%%%% Pre-document %%%%%%%%%%%%%%%%%%%%%
\usepackage{fancyhdr}
\usepackage{titlesec}
\usepackage{float}
\usepackage{array}
\usepackage{nicematrix}
\usepackage{multicol}
\usepackage{tikz}
\usetikzlibrary{arrows,positioning, calc, arrows.meta}
\usepackage{enumitem}

\setlength{\parindent}{0pt} % Remove auto paragraph indents
\renewcommand\labelitemi{{\boldmath$\cdot$}}
\setlength{\mathindent}{0pt}  

% Get rid of those big margins
\usepackage[margin=1.5in]{geometry}
\newlength\titleindent
\setlength\titleindent{2cm}

\titleformat{\section}
  {\large\bfseries}{\llap{\parbox{\titleindent}{\bfseries\thesection\hfill}}}{0em}{}
\titleformat{\subsection}
  {\large}{\llap{\parbox{\titleindent}{\bfseries\thesubsection\hfill}}}{0em}{\bfseries}
\titleformat{\subsubsection}
  {\large}{\llap{\parbox{\titleindent}{\bfseries\thesubsubsection}}}{0em}{\bfseries}

\begin{document}

\pagestyle{fancy}
% Header
\fancyhead{}
\fancyhead[L]{Stephanie L'Heureux}
\fancyhead[R]{\thepage}
% No page numbers for footer
\fancyfoot{}

\begin{center}
    \Large{\textbf{Chapter 1 Exercises}}\\
\end{center}

\setcounter{section}{1}
\setcounter{subsection}{10}

\subsection{Assume a 15cm diameter wafer has a cost of 12, contains 84 dies, and has 0.020 defects/cm$^2$. Assume a 20cm diameter wafer has a cost of 15, contains 100 dies, and has 0.031 defects/cm$^2$.}
\subsubsection{Find the yield for both wafers.}
\[ \text{yield} = \frac{1}{(1 + (\text{Defects per area} \times \text{Die area}))^N} \]

\textbf{\\Wafer 1:}
\begin{itemize}[leftmargin=*]
    \item $\text{Wafer area}_1 = \pi r ^2 = \pi (\frac{15}{2})^2 = \frac{\pi 225}{4}$
    \item $\text{Die area}_1 = \frac{\text{Wafer area}}{\text{Dies per wafer}} = \frac{(\pi225)/4}{84} = \frac{\pi75}{112}$
    \item $\text{Defects per area}_1 = 0.020$
    \item $N_1 = 2$
\end{itemize}
\[ \text{yield}_1 = \frac{1}{(1 + (0.020 \times \frac{\pi75}{112}))^2} = 0.9208... \Rightarrow 0.92\]
\[ \boxed{\text{yield}_1 = 0.92}\]

\textbf{\\Wafer 2:}
\begin{itemize}[leftmargin=*]
    \item $\text{Wafer area}_2 = \pi r ^2 = \pi (\frac{20cm}{2})^2 = \pi100$
    \item $\text{Die area}_2 = \frac{\text{Wafer area}}{\text{Dies per wafer}} = \frac{\pi100}{100} = \pi $
    \item $\text{Defects per area}_2 = 0.031$
    \item $N_2 = 2$
\end{itemize}
\[ \text{yield}_2 = \frac{1}{(1 + (0.031 \times \pi ))^2} = 0.8303... \Rightarrow 0.83 \]
\[ \boxed{\text{yield}_2 = 0.83} \]


\subsubsection{Find the cost per die for both wafers.}
\[ \text{Cost per die} = \frac{\text{Cost per wafer}}{\text{Dies per wafer} \times \text{yield}} \]

\textbf{\\Wafer 1:}
\begin{itemize}[leftmargin=*]
    \item $\text{Cost per wafer}_1 = 12$
    \item $\text{Dies per wafer}_1 = 84$
    \item $\text{yeild}_1 = 0.9208$
\end{itemize}
\[ \text{Cost per die}_1 = \frac{12}{84 \times 0.9208} = 0.1551... \Rightarrow 0.16 \]
\[ \boxed{\text{Cost per die}_1 = 0.16} \]


\textbf{\\Wafer 2:}
\begin{itemize}[leftmargin=*]
    \item $\text{Cost per wafer}_2 = 15$
    \item $\text{Dies per wafer}_2 = 100$
    \item $\text{yeild}_2 = 0.8303$
\end{itemize}
\[ \text{Cost per die}_2 = \frac{15}{100 \times 0.8303} = 0.1806... \Rightarrow 0.18 \]
\[ \boxed{\text{Cost per die}_2 = 0.18} \]

\subsubsection{If the number of dies per wafer is increased by 10\% and the defects per area unit increases by 15\%, find the die area and yeild.}
\textbf{\\Wafer 1:}
\begin{itemize}[leftmargin=*]
    \item $\text{Dies per wafer}_1 = 84 + (0.1 \times 84) = 92.4$
    \item $\text{Wafer area}_1 = \pi r ^2 = \pi (\frac{15cm}{2})^2 = \frac{\pi 225}{4}$
    \item $\text{Defects per area}_1 = 0.020 + (0.15 \times 0.020) = 0.023$
    \item $N_1 = 2$
    \item $\text{Cost per wafer}_1 = 12$
\end{itemize}

\[ \text{Die area}_1 = \frac{\text{Wafer area}}{\text{Dies per wafer}} = \frac{(\pi225)/4}{92.4} = \frac{\pi375}{616} \]
\[ \boxed{\text{Die area}_1 = \frac{\pi375}{616}} \]

\[ \text{yield}_1 = \frac{1}{(1 + (0.023 \times \frac{\pi375}{616}))^2} = 0.9175... \Rightarrow 0.92\]
\[ \boxed{\text{yield}_1 = 0.92}\]


\textbf{\\Wafer 2:}
\begin{itemize}[leftmargin=*]
    \item $\text{Dies per wafer}_2 = 100 + (0.1 \times 100) = 110$
    \item $\text{Wafer area}_2 = \pi r ^2 = \pi (\frac{20cm}{2})^2 = \pi100$
    \item $\text{Defects per area}_2 = 0.031 + (0.15 \times 0.031) = 0.03565$
    \item $N_2 = 2$
    \item $\text{Cost per wafer}_2 = 15$
\end{itemize}
\[ \text{Die area}_2 = \frac{\text{Wafer area}}{\text{Dies per wafer}} = \frac{\pi100}{110} = \frac{\pi10}{11} \]
\[ \boxed{\text{Die area}_2 = \frac{\pi10}{11}} \]

\[ \text{yield}_2 = \frac{1}{(1 + (0.03565 \times \frac{\pi10}{11}))^2} = 0.8237... \Rightarrow 0.82 \]
\[ \boxed{\text{yield}_2 = 0.82} \]

\pagebreak
\setcounter{subsection}{14}
\subsection{Assume a program requires the execution of $50 \times 10^6$ FP instructions. $110 \times 10^6$ INT instructions, $80 \times 10^6$ L/S instructions, and $10 \times 10^6$ branch instructions. The CPI for each type of instruction is 1, 1, 4, and 2, respectively. Assume the processor has 2GHz clock rate.}

\subsubsection{By how much must we improve the CPI of FP instructions if we want the program to run two times faster?}
Total CPU clock cycles before optimization:
\[ (50 \times 10^6)(1) + (110 \times 10^6)(1) + (80 \times 10^6)(4) + (16 \times 10^6)(2) =  512 \times 10^6 \]
Total CPU execution time before optimization:
\[ \frac{ 512 \times 10^6}{2 \text{GHz}} = 256 \times 10^{-3} \text{s}\]
Goal CPU execution time before optimization:
\[ \frac{256 \times 10^{-3} \text{s}}{2} = 128 \times 10^{-3} \text{s}\]
Goal total cycles for FP:
\[128 \times 10^{-3} \text{s} \times 2\text{GHz} =  256 \times 10^6\]
FP CPI:
\[ \text{FP CPI} =  \frac{256 - 462}{50} \]
\[ \text{FP CPI} = -4.12 \]
\[ \boxed{\text{It is not possible to improve the FP CPI to make the program run 2 times faster.}} \]


\subsubsection{By how much must we improve the CPI of L/S instructions if we want the program to run two times faster?}
Goal CPU execution time before optimization:
\[ \ \frac{256 \times 10^{-3} \text{s}}{2} = 128 \times 10^{-3} \text{s}\]
Goal total cycles for L/S:
\[ 128 \times 10^{-3} \text{s} \times 2\text{GHz} =  256 \times 10^6\]

L/S CPI:
\[ (50 \times 10^6)(1) + (110 \times 10^6)(1) + (80 \times 10^6)(\text{L/S CPI}) + (16 \times 10^6)(2) =  512 \times 10^6 \]
\[ \text{L/S CPI} =  \frac{256 - 192}{80} \]
\[ \text{FP CPI} = 4.625 \]

CPI Improvement: 
\[ 4 / 0.8 = 5 \]
\[ \boxed{\text{L/S instructions must be 5 times faster}} \]


\subsubsection{By how much is the execution time of the program improved if the CPI of INT and FP instructions is reduced by 40\% and the CPI of L/S and Branch is reduced by 30\%?}
New CPI
\begin{itemize}[leftmargin=*]
    \item FP: $(1 - (1) \times 0.4) = 0.6$
    \item INT:  $(1 - (1) \times 0.4) = 0.6$
    \item L/S: $(4 - (4) \times 0.3)= 2.8$
    \item Branch: $(2 - (2) \times 0.3) = 1.4$
\end{itemize}

New total CPU clock cycles
\[ (50 \times 10^6)(0.6) + (110 \times 10^6)(0.6) + (80 \times 10^6)(2.8) + (16 \times 10^6)(1.4) =  342.4\times 10^6 \]

New CPU execution time
\[ \frac{342.4\times 10^6}{2\text{GHz}} = 171.2 \times 10^{-3}\text{s}\] 

CPU execution time improvement:
\[ \frac{256 \times 10^{-3}}  {171.2 \text{s} \times 10^{-3} \text{s}} \times 100\% = 149.5 \% \]
\[ \boxed{\text{The execution time is improved by } 149.5 \% } \]
\end{document}