\documentclass[fleqn]{article}

%%%%%%%%%%%%%%%%%%%%% Pre-document %%%%%%%%%%%%%%%%%%%%%
\usepackage{fancyhdr}
\usepackage{titlesec}
\usepackage{float}
\usepackage{array}
\usepackage{nicematrix}
\usepackage{multicol}
\usepackage{enumitem}
\usepackage{listings}
\usepackage{xcolor}
\usepackage{scrextend}

\lstdefinelanguage[RISC-V]{Assembler} {
  alsoletter={.},
  alsodigit={0x},
  morekeywords=[1]{ % instructions
    lw, sw,
    sll, slli,
    add, addi, sub,
    xor, xori, or, ori, and, andi,
    beq, bne, blt, bge, bltu, bgeu,
    j, jr, jal, jalr, ret,
  },
  morekeywords=[2]{ % registers
    x0, x1, x2, x3, x4, x5, x6, x7, x7, x8, x9, x10,
    x11, x12, x13, x14, x15, x16, x17, x18, x19, x20, 
    x21, x22, x23, x24, x25, x26, x27, x28, x29, x30, x31 },
  morecomment=[l]{;},   % mark ; as line comment start
  morecomment=[l]{\#},  % as well as #
}

% define some colors
\definecolor{pink}{rgb}{204, 0, 68}
\definecolor{blue}{rgb}{77, 148, 255}

\lstset{
  % listings sonderzeichen (for german weirdness)
  literate={ö}{{\"o}}1
           {ä}{{\"a}}1
           {ü}{{\"u}}1,
  basicstyle=\ttfamily,
  breaklines=true,
  commentstyle=\color{gray!50!black},
  keywordstyle=[1]\color{pink},
  % keywordstyle=[2]\color{blue}, 
  stringstyle=\color{mauve},                    % strings are from the telekom
  language=[RISC-V]Assembler,                   % all code is RISC-V
  tabsize=4,                                    % indent tabs with 4 spaces
  showstringspaces=false                        % do not replace spaces with weird underlines
}


\setlength{\parindent}{0pt} % Remove auto paragraph indents
\renewcommand\labelitemi{{\boldmath$\cdot$}}
\setlength{\mathindent}{0pt}  

% Get rid of those big margins
\usepackage[margin=1in]{geometry}
\newlength\titleindent
\setlength\titleindent{2cm}

\begin{document}

\pagestyle{fancy}
% Header
\fancyhead{}
\fancyhead[L]{Stephanie L'Heureux}
\fancyhead[R]{\thepage}
% No page numbers for footer
\fancyfoot{}

\begin{center}
    \Large{\textbf{Chapter 3 Exercises}}\\
\end{center}
\vspace{0.25in}

\textbf{3.3} [10] \textlangle 3.2\textrangle \; Convert 5ED4 into a binary number. What makes base 16 (hexadecimal) an attractive numbering system for representing values in computers?
\vspace{0.125in}

\begin{addmargin}[0.15cm]{0cm}
\begin{table}[H]
    \begin{tabular}{l|llll}
        hex & 5 & E & D & 4 \\
        ten & 5 & 14 & 13 & 4 \\
        two & 0100 & 1110 & 1101 & 0100
    \end{tabular}
\end{table}

5ED4\textsubscript{hex} is 0100 1110 1101 0100\textsubscript{two}. 
Hexadecimal is the base-16 number system, a numeric representation format in which each digit can take on 16 values, ranging from 0 to F. Each hexadecimal digit therefore corresponds directly to 4 bits, making it convenient for representing binary values, as the translation between the two systems is straightforward. Hexadecimal is particularly useful in computing applications. It offers a more concise format than binary while maintaining easy conversion between the two number systems.
\end{addmargin}

\vspace{0.5in}
\textbf{3.20} [5] \textlangle 3.5\textrangle \; What decimal number does the bit pattern 0x0C000000 represent if it is a two's complement integer? An unsigned integer?
\vspace{0.125in}

\begin{addmargin}[0.15cm]{0cm}
\begin{table}[H]
    \begin{tabular}{l|llllllll}
        & $16^7$ & $16^6$ & $16^5$ & $16^4$ & $16^3$ & $16^2$ & $16^1$ & $16^0$ \\ 
        hex & 0 & C & 0 & 0 & 0 & 0 & 0 & 0 \\
        two & 0000 & 1100 & 0000 & 0000 & 0000 & 0000 & 0000 & 0000
    \end{tabular}
\end{table}
$12 \times 16^6 = 201,326,592$

\vspace{0.125in}
Two's complement integer: $201,326,592_{\text{ten}}$

Unsigned integer: $201,326,592_{\text{ten}}$

\end{addmargin}

\vspace{0.5in}
\textbf{3.21} [10] \textlangle 3.5\textrangle \; If the bit pattern 0x0000006F is placed into the Instruction Register, what RISC-V instruction will be executed?
\begin{addmargin}[0.15cm]{0cm}
\begin{table}[H]
    \begin{tabular}{l|llllllll}
        & $16^7$ & $16^6$ & $16^5$ & $16^4$ & $16^3$ & $16^2$ & $16^1$ & $16^0$ \\ 
        hex & 0 & 0 & 0 & 0 & 0 & 0 & 6 & F \\
        two & 0000 & 0000 & 0000 & 0000 & 0000 & 0000 & 0110 & 1111
    \end{tabular}
\end{table}
The last 7 bits corresponds to the optcode of the instruction. The optcode denotes the the operation and format of the instruction. It can be determined the instruction is \verb|jal x0,0(0)|, as \verb|jal| is the only RISC-V instruction matching the optcode 1101111\textsubscript{two}.

\end{addmargin}

\vspace{0.5in}
\textbf{3.22} [10] \textlangle 3.5\textrangle \; What decimal number does the bit pattern 0x0C000000 represent if it is a floating point number? Use the IEEE 754 standard.
\begin{addmargin}[0.15cm]{0cm}
\begin{table}[H]
    \setlength{\tabcolsep}{4pt}
    \fontsize{7pt}{8pt}\selectfont
    \begin{tabular}{c|cccccccc|ccccccccccccccccccccccc}
        s & \multicolumn{8}{c|}{exponent} & \multicolumn{23}{c}{fraction} \\
        31 & 30 & 29 & 28 & 27 & 26 & 25 & 24 & 23 & 22 & 21 & 20 & 19 & 18 & 17 & 16 & 15 & 14 & 13 & 12 & 11 & 10 & 9 & 8 & 7 & 6 & 5 & 4 & 3 & 2 & 1 & 0 \\
        0 & 0 & 0 & 0 & 1 & 1 & 0 & 0 & 0 & 0 & 0 & 0 & 0 & 0 & 0 & 0 & 0 & 0 & 0 & 0 & 0 & 0 & 0 & 0 & 0 & 0 & 0 & 0 & 0 & 0 & 0 & 0
    \end{tabular}
\end{table}
\[ (-1)^s \times (1 + \text{Fraction}) \times  2^{\text{Exponent - Bias}} = (-1)^0 \times (1 + 0) \times 2^{(24 - 127)} = 2^{-103} = 9.86076132 \times 10^{-32}_{\text{ten}}\]
\end{addmargin}

\vspace{0.5in}
\textbf{3.23} [10] \textlangle 3.5\textrangle \; Write down the binary representation of the decimal number 63.25 assuming the IEEE 754 single precision format.
\begin{addmargin}[0.15cm]{0cm}
\[63.25_{\text{ten}} = \frac{253}{4}_{\text{ten}} = \frac{253}{2^2}_{\text{ten}} \Rightarrow 1111 11.01_{\text{two}} \times 2^0 \Rightarrow 1.111 1101_{\text{two}} \times 2^5 \]
\begin{table}[H]
    \setlength{\tabcolsep}{4pt}
    \fontsize{7pt}{8pt}\selectfont
    \begin{tabular}{c|cccccccc|ccccccccccccccccccccccc}
        s & \multicolumn{8}{c|}{exponent} & \multicolumn{23}{c}{fraction} \\
        31 & 30 & 29 & 28 & 27 & 26 & 25 & 24 & 23 & 22 & 21 & 20 & 19 & 18 & 17 & 16 & 15 & 14 & 13 & 12 & 11 & 10 & 9 & 8 & 7 & 6 & 5 & 4 & 3 & 2 & 1 & 0 \\
        0 & 1 & 0 & 0 & 0 & 0 & 1 & 0 & 0 & 1 & 1 & 1 & 1 & 1 & 0 & 1 & 0 & 0 & 0 & 0 & 0 & 0 & 0 & 0 & 0 & 0 & 0 & 0 & 0 & 0 & 0 & 0
    \end{tabular}
\end{table}
63.25\textsubscript{ten} = 0100 0010 0111 1101 0000 0000 0000 0000\textsubscript{two}
\end{addmargin}

\end{document} 