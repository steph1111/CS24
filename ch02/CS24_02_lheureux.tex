\documentclass[fleqn]{article}

%%%%%%%%%%%%%%%%%%%%% Pre-document %%%%%%%%%%%%%%%%%%%%%
\usepackage{fancyhdr}
\usepackage{titlesec}
\usepackage{float}
\usepackage{array}
\usepackage{nicematrix}
\usepackage{multicol}
\usepackage{enumitem}
\usepackage{listings}
\usepackage{xcolor}


\lstdefinelanguage[RISC-V]{Assembler} {
  alsoletter={.},
  alsodigit={0x},
  morekeywords=[1]{ % instructions
    lw, sw,
    sll, slli,
    add, addi, sub,
    xor, xori, or, ori, and, andi,
    beq, bne, blt, bge, bltu, bgeu,
    j, jr, jal, jalr, ret,
  },
  morekeywords=[2]{ % registers
    x0, x1, x2, x3, x4, x5, x6, x7, x7, x8, x9, x10,
    x11, x12, x13, x14, x15, x16, x17, x18, x19, x20, 
    x21, x22, x23, x24, x25, x26, x27, x28, x29, x30, x31 },
  morecomment=[l]{;},   % mark ; as line comment start
  morecomment=[l]{\#},  % as well as #
}

% define some colors
\definecolor{pink}{rgb}{204, 0, 68}
\definecolor{blue}{rgb}{77, 148, 255}

\lstset{
  % listings sonderzeichen (for german weirdness)
  literate={ö}{{\"o}}1
           {ä}{{\"a}}1
           {ü}{{\"u}}1,
  basicstyle=\ttfamily,
  breaklines=true,
  commentstyle=\color{gray!50!black},
  keywordstyle=[1]\color{pink},
  % keywordstyle=[2]\color{blue}, 
  stringstyle=\color{mauve},                    % strings are from the telekom
  language=[RISC-V]Assembler,                   % all code is RISC-V
  tabsize=4,                                    % indent tabs with 4 spaces
  showstringspaces=false                        % do not replace spaces with weird underlines
}


\setlength{\parindent}{0pt} % Remove auto paragraph indents
\renewcommand\labelitemi{{\boldmath$\cdot$}}
\setlength{\mathindent}{0pt}  

% Get rid of those big margins
\usepackage[margin=1in]{geometry}
\newlength\titleindent
\setlength\titleindent{2cm}

\begin{document}

\pagestyle{fancy}
% Header
\fancyhead{}
\fancyhead[L]{Stephanie L'Heureux}
\fancyhead[R]{\thepage}
% No page numbers for footer
\fancyfoot{}

\begin{center}
    \Large{\textbf{Chapter 2 Exercises}}\\
\end{center}
\vspace{0.25in}

\textbf{2.3} [5] \textlangle 2.2\textrangle \; For the following C statement, write the corresponding RISC-V assembly code. Assume that the variables \verb|f|, \verb|g|, \verb|h|, \verb|i|, and \verb|j| are assigned to registers \verb|x5|, \verb|x6|, \verb|x7|, \verb|x28|, and \verb|x29|, respectively. Assume that the base address of the arrays \verb|A| and \verb|B| are in registers \verb|x10| and \verb|x11|, respectively.

\begin{verbatim}
    B[8] = A[i-j];
\end{verbatim}

RISC-V Translation:
\begin{lstlisting}
    sub x5 x28, x29     # f = i - j
    slli x5, x5, 2      # f = f << 2     (f = f * 4) 
    add x6, x10, x5     # g = &A[i-j]
    lw x6, 0(x6)        # g = A[i-j]
    sw x6, 32(x11)      # B[8] = A[i-j]  (8 * 4 = 32)
\end{lstlisting}

\vspace{0.5in}

\textbf{2.7} [5] \textlangle 2.2, 2.3\textrangle \; Translate the following C code to RISC-V. Assume that the variables \verb|f|, \verb|g|, \verb|h|, \verb|i|, and \verb|j| are assigned to registers \verb|x5|, \verb|x6|, \verb|x7|, \verb|x28|, and \verb|x29|, respectively. Assume that the base address of the arrays \verb|A| and \verb|B| are in registers \verb|x10| and \verb|x11|, respectively. Assume that the elements of the arrays \verb|A| and \verb|B| are 8-byte words:
\begin{verbatim}
    B[8] = A[i] + A[j];
\end{verbatim}

RISC-V Translation:
\begin{lstlisting}
    slli x28, x28, 2    # i = i << 2
    add x5, x10, x28    # f = &A[i]
    lw x5, 0(x5)        # f = A[i]
    slli x29, x29, 2    # j = j << 2
    add x6, x10, x29    # g = &A[j]
    lw x6. 0(x6)        # g = A[j]
    add x7, x5, x6      # h = f + g
    sw x7, 32(x11)      # B[8] = h
\end{lstlisting}

\pagebreak

\textbf{2.13} [5] \textlangle 2.5\textrangle \; Provide the instruction type and hexadecimal representation for the following instruction:
\begin{lstlisting}
    sw x5, 32(x30)
\end{lstlisting}
\vspace{0.25in}
Instruction format S.
\[0000001 \; 00101 \; 11110 \; 010  \; 00000 \; 0100011_{\text{two}} \Rightarrow 25F2023_{\text{hex}} \]

\vspace{0.5in}


\textbf{2.18} [10] \textlangle 2.6\textrangle \; Find the shortest sequence of RISC-V instructions that extracts bits 16 down to 11 from register \verb|x5| and uses the value of this field to replace bits 31 down to 26 in register \verb|x6| without changing the other bits of registers \verb|x5| or \verb|x6|. (Be sure to test your code using \verb|x5 = 0| and \verb|x6 = Oxffffffffffffffff|. Doing so may reveal a common oversight.)
\begin{lstlisting}
    andi x7, x5, 0x1F800        # Extract bits 16 to 11
    slli x7, x7, 15             # Shift bits 16 to 11 to position 31 to 26 
    andi x6, x6, 0x3FFFFFF      # Clear bits 31 to 26 
    or x6, x6, x7               # Replace bits
\end{lstlisting}

\end{document} 